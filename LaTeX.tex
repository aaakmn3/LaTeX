\documentclass{article}
\usepackage[utf8]{inputenc}
\usepackage{amsmath}
\usepackage{xcolor}
\usepackage{mathtools}  

\title{LaTeX (Matematik)}
\author{Ahmet Alp AKMAN}
\date{August 27, 2022}

\begin{document}

\maketitle


\section{Getting Started}

\textbf{LaTeX (Matematik)} Today I am learning \LaTeX{}. \LaTeX{} is a great program for writing math. I can write in line math such as $a^{2}+b^{2}=c^{2}$. I can also give equations their own space:   
\begin{equation} \label{eq:eps}
\gamma^{2}+\theta^{2}=\omega^{2} 
\end{equation}\\`` Maxwell's equations'' are named for James Clark Maxwell and are as follow:\\

\begin{equation} \label{eq:eps1}
\hspace{-60}
\vec{\nabla} \cdot \vec{E}  \hspace{5} \textbf{=}\hspace{5} \frac{\rho}{\varepsilon_0}\hspace{100} \textrm{Gauss's Law}
\end{equation}
\begin{equation} \label{eq:eps2}
\hspace{0}                 
\vec{\nabla} \cdot \vec{B} \hspace{5} \textbf{=}\hspace{5}0  \hspace{100} \textrm{Gauss's Law for Magnetism}
\end{equation}
\begin{equation} \label{eq:eps3}
\hspace{-5} \vec{\nabla}  \times \vec{E}  \hspace{5} \textbf{=}\hspace{5}- \frac{\partial \vec{B}}{\partial t}\hspace{77}
\textrm{Faraday's Law of Induction}
\end{equation}
\begin{equation} \label{eq:eps4}
\hspace{-24}\vec{\nabla} \times \vec{B}  \hspace{5} \textbf{=}\hspace{5}  \mu_0 \left( \varepsilon_0\frac{\partial \vec{E}}{\partial t}+\vec{J} \right)\hspace{37} \textrm{Ampere's Circuital Law}
\end{equation}
 Denklem \textcolor{blue}{2, 3, 4} and \textcolor{blue}{5}   are some of the most important in Physics.

\section{What about Matrix Equations?}
\begin{equation*}
\begin{pmatrix}
a_{11} & a_{12} & \cdots & a_{1n} \\
a_{21} & a_{22} & \cdots & a_{2n} \\
\vdots  & \vdots  & \ddots & \vdots  \\
a_{n1} & a_{n2} & \cdots & a_{nn} 
\end{pmatrix}
%
  \begin{bmatrix}
    v_{1} \\
    v_{2} \\
    \vdots\\
    v_{n} \\
  \end{bmatrix} 
% 
=
\begin{matrix}
    w_{1} \\
    w_{2} \\
    \vdots\\
    w_{n} \\
  \end{matrix}
\end{equation*}

 \pagebreak

   

\begin{displaymath}
  \iiint\limits_V f(x,y,z) \, dV =F  
\end{displaymath}

\begin{displaymath}
\frac{dx}{dy} = x'= \lim\limits_{h \to 0} \frac{f(x + h )-f(x)}{h}= 0  
\end{displaymath} 

 \begin{displaymath}
|x| = \left\{ \begin{array}{ll}
-x & if \textrm{ $x<0$ ise}\\
x & if \textrm{ $x \geq 0$ ise}\\
\end{array} \right.
\end{displaymath}
 
\begin{displaymath}
F(x) = A_{0}+
\sum_{n=1}^{N} \begin{bmatrix}
  A_{n}\cos\Big(\frac{2\pi n}{P}\Big)+
  B_{n}\sin\Big(\frac{2\pi n}{P}\Big)
  \end{bmatrix} 
\end{displaymath}

\begin{displaymath}
\sum_{n=1} \frac{1}{n^{s}} = \prod_P \frac{1}{1-\frac{1}{P^{n}}}
\end{displaymath} 
 
 \begin{displaymath}
 m\ddot{x}+c\dot{x}+kx=F_{0}\sin(2\pi ft)
 \end{displaymath}
 
\begin{displaymath}f(x)&\hspace{5}=\hspace{5}x^2 +3x+ 5x^2+8+6x\\
   
 \hspace{109} &=\hspace{5}6x^2+9x+8\\
 
 \hspace{109}  =\hspace{5}x(6x+9)+8
\end{displaymath}
\begin{displaymath}
X=\frac{F_0}{k}\frac{1}{\sqrt{(1-r^2)^2+(2 \zeta r^2)}}
\end{displaymath}

\begin{displaymath}
G_{\mu\upsilon}\equiv R_{\mu\upsilon} - \frac{1}{2}Rg_{\mu\upsilon}=\frac{8\pi G}{c^4}T_{\mu\upsilon}
\end{displaymath}\\



\begin{displaymath}
6CO_2+6H_2O \to C_6H_12O_6 + 6O_2
\end{displaymath}

\begin{displaymath}
SO{_4}{^2^-}+Ba^2^+ \to BaSO_4
\end{displaymath}

\begin{equation*}
\begin{pmatrix}
a_{11} & a_{12} & \cdots & a_{1n} \\
a_{21} & a_{22} & \cdots & a_{2n} \\
\vdots  & \vdots  & \ddots & \vdots  \\
a_{n1} & a_{n2} & \cdots & a_{nn} 
\end{pmatrix}
%
  \begin{bmatrix}
    v_{1} \\
    v_{2} \\
    \vdots\\
    v_{n} \\
  \end{bmatrix} 
% 
=
\begin{matrix}
    w_{1} \\
    w_{2} \\
    \vdots\\
    w_{n} \\
  \end{matrix}
\end{equation*}
\begin{displaymath}
\frac{\partial_u}{\partial_t} + (u . \nabla)u-\upsilon\nabla^2(u)=-\nabla h
\end{displaymath}

\begin{displaymath}
\alpha A \beta B \gamma \Gamma \delta \Delta \pi \Pi \omega \Omega 
\end{displaymath}




\end{document}